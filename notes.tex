\documentclass{article}

%----------------------------------------------------------------------------------------

\usepackage{listings} % Required for inserting code snippets
\usepackage[usenames,dvipsnames]{color} % Required for specifying custom colors and referring to colors by name
\usepackage{courier}

\definecolor{DarkGreen}{rgb}{0.0,0.4,0.0} % Comment color
\definecolor{highlight}{RGB}{255,251,204} % Code highlight color

\lstdefinestyle{Style1}{ % Define a style for your code snippet, multiple definitions can be made if, for example, you wish to insert multiple code snippets using different programming languages into one document
language=Java, % Detects keywords, comments, strings, functions, etc for the language specified
%backgroundcolor=\color{highlight}, % Set the background color for the snippet - useful for highlighting
basicstyle=\footnotesize\ttfamily, % The default font size and style of the code
breakatwhitespace=false, % If true, only allows line breaks at white space
breaklines=true, % Automatic line breaking (prevents code from protruding outside the box)
captionpos=b, % Sets the caption position: b for bottom; t for top
commentstyle=\usefont{T1}{pcr}{m}{sl}\color{DarkGreen}, % Style of comments within the code - dark green courier font
deletekeywords={}, % If you want to delete any keywords from the current language separate them by commas
%escapeinside={\%}, % This allows you to escape to LaTeX using the character in the bracket
firstnumber=1, % Line numbers begin at line 1
frame=single, % Frame around the code box, value can be: none, leftline, topline, bottomline, lines, single, shadowbox
frameround=tttt, % Rounds the corners of the frame for the top left, top right, bottom left and bottom right positions
keywordstyle=\color{Blue}\bf, % Functions are bold and blue
morekeywords={}, % Add any functions no included by default here separated by commas
numbers=left, % Location of line numbers, can take the values of: none, left, right
numbersep=10pt, % Distance of line numbers from the code box
numberstyle=\tiny\color{Gray}, % Style used for line numbers
rulecolor=\color{black}, % Frame border color
showstringspaces=false, % Don't put marks in string spaces
showtabs=false, % Display tabs in the code as lines
stepnumber=5, % The step distance between line numbers, i.e. how often will lines be numbered
stringstyle=\color{Purple}, % Strings are purple
tabsize=2, % Number of spaces per tab in the code
}

\newcommand{\insertcode}[2]{\begin{itemize}\item[]\lstinputlisting[caption=#2,label=#1,style=Style1]{#1}\end{itemize}} % 

\begin{document}

\section{Summary}
\subsection{Lost messages}
\subsubsection{RMI}
No messages were lost with RMI.
\subsubsection{UDP}
One probable cause is that the receiving buffer gets full, not accepting anymore messages until 
it clears space in the buffer. Another probable cause is that it is lost in transit, although it 
is not very likely because we are testing in a LAN. Finally, it may be the case that the sending
buffer gets filled up when sending a lot of packets, thus dropping some packets when filled.
\subsection{Patterns in lost messages}
\subsubsection{RMI}
No messages were lost with RMI.
\subsubsection{UDP}
The general pattern with UDP is that no messages are lost for a few client connections in a row,
and suddenly approximately 40\% of the messages are lost in the next connection. The cycle then repeats.
\subsection{Reliability}
\subsubsection{RMI}
RMI is very reliable. In our case, we haven't seen a single message lost.
\subsubsection{UDP}
UDP is very unreliable. Although it is true that sometimes no messages are lost, it might be the
case that the next connection loses half of the messages sent, affecting severely its reliability.
\subsection{Difficulty to program}
In our case, we found it easier to program the RMI client and server. This was due to the fact
that we didn't have to start from scratch because \texttt{RMIServer} extends \texttt{UnicastRemoteObject} so 
a lot of functionality is hidden from us. On the other hand, with UDP we don't have a Registry
so we have to handle connections in a lower level.
\section{Console logs}

\section{Listings}
\subsection{RMI Client}
\insertcode{"rmi/RMIClient.java"}{RMI Client}

\subsection{RMI Server}
\insertcode{"rmi/RMIServer.java"}{RMI Server}

\subsection{UDP Client}
\insertcode{"udp/UDPClient.java"}{UDP Client}

\subsection{UDP Server}
\insertcode{"udp/UDPServer.java"}{UDP Server}


\end{document}